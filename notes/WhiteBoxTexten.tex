\mysubsubsection{White-Box Testing}

%%%%%%%%%%%%%%%%%%%%%%%%%%%%%%%%%

\begin{frame}
\frametitle{White-box Testing of Modules}
\framesubtitle{(testing in the small)}
\begin{description}[Anwei]
  \item[C0: Statement Coverage:] All executable statements in the source code are executed at least once (Statement coverage criterion).
  \item[C1: Edge coverage:] Each edge of the control graph being traversed at least once (Edge-coverage criterion).
  \item[C2: Path (Branch) Coverage:] All paths leading from the initial to the final node of the control graph being traversed (Path-coverage criterion).
  \item[C3: (Compound) Condition Coverage:] Each possible combination of conditions must be executed at least once (Condition coverage criterion).
\end{description}
\end{frame}
%https://www.cs.colostate.edu/~bieman/CS314/Notes/GhoshFranceStructuralTesting.pdf
%http://www.cas.mcmaster.ca/khedri/wp-content/uploads/COURSES/3S03/Tutorial04.pdf

%%%%%%%%%%%%%%%%%%%%%%%%%%%%%%%%%

\begin{frame}
\frametitle{Problems with Statement coverage}
\framesubtitle{C0: Statement coverage}
\begin{itemize}
  \item Is a program sufficiently tested when each statement has been executed at least once?
  \item How to determine a minimum test quantity? 
  \item Is it useful to cover even empty statements (e.g. missing \texttt{else})? 
  \item The structure of the program is not taken into account 
  \item Basically:\\
        Coverage is not decisive!
\end{itemize}
\end{frame}

%%%%%%%%%%%%%%%%%%%%%%%%%%%%%%%%%

\begin{frame}
\frametitle{Coverage of all edges of a control flow graph}
\framesubtitle{C1: Edge-coverage}
Precondition is the construction of a control flow graph for an (imperative) program:
\begin{itemize}
  \item Every single statement, which does not contain any further statements, is represented as a node.
  \item Conditional statements are represented as branches.
  \item Loops are represented as cycles and 
  \item Sequences by edges between nodes.
\end{itemize}
\end{frame}

%%%%%%%%%%%%%%%%%%%%%%%%%%%%%%%%%

\begin{frame}
\frametitle{Construction of control flow graphs}
\begin{center}
\pgfimage[width=\textwidth]{QualityControl/illustrations/ControlFlowGraphs}
\end{center}
\end{frame}

%%%%%%%%%%%%%%%%%%%%%%%%%%%%%%%%%

\begin{frame}[fragile]
\frametitle{Example: Program for faculty computation}
\begin{columns}
\begin{column}{0.4\textwidth}
\lstset{language=Java,basicstyle=\small}
\begin{lstlisting}
public long faculty
            (int n) {
  if (n < 0) {
    throw new 
      Exception();
  }
  long fac = 1;

  while (n > 0) {
    fac *= n;
    n--;
   }
  return fac;
}
\end{lstlisting}
\end{column}
\pause
\begin{column}{0.6\textwidth}
\begin{center}
\pgfimage[width=\textwidth]{QualityControl/illustrations/FacultyCalculation}
\end{center}
\end{column}
\end{columns}
\end{frame}

%%%%%%%%%%%%%%%%%%%%%%%%%%%%%%%%%

\begin{frame}
\frametitle{Test cases for Edge-coverage}
\begin{tabular}{llp{0.2\textwidth}p{0.18\textwidth}p{0.28\textwidth}}
\textbf{Nr} & \textbf{Class} & \textbf{Test case description} & \textbf{Expected Results}  & \textbf{Exemplary input data} \\
~\\
1           & Normal          & Input parameter is valid	& Faculty of the input parameter	& 42 \\
            & \multicolumn{4}{l}{Checked edges: 2, 3, 4, 5, 6, 7} \\[2em]
2           & Error          & Input parameter is invalid   & Exception & -1 \\
            & \multicolumn{4}{l}{Checked edges: 1}
\end{tabular}
\end{frame}

%%%%%%%

\begin{frame}
\frametitle{Cyclomatic Complexity}
\framesubtitle{\citet{McCabe1976,EbertCain2016}}
\begin{itemize}
  \item Behind this software metric by McCabe is the idea that above a certain complexity a module is no longer comprehensible to humans. 
  \item The cyclomatic complexity is defined as the number of independent paths on the control flow graph of a module.
  \item The cyclomatic complexity indicates how many test cases are needed to achieve Edge-coverage.
  \item The complexity measure according to McCabe is equal to the number of binary branches plus 1.
  \item According to McCabe, the cyclomatic number of a self-contained subprogram should not exceed 10, otherwise the program is too complex and too difficult to test.
\end{itemize}
\end{frame}


\begin{frame}
\frametitle{Cyclomatic Complexity: Example}
\framesubtitle{\citet{McCabe1976,EbertCain2016}}
\begin{columns}
\begin{column}{0.5\textwidth}
\begin{center}
\pgfimage[width=\textwidth]{QualityControl/illustrations/CylomaticComplexity}
\end{center}
\end{column}
\begin{column}{0.5\textwidth}
\pause
Path 1:  1,2,3,6,7,8\\
\pause
Path 2:  1,2,3,5,7,8\\
\pause
Path 3:  1,2,4,7,8\\
\pause
Path 4:  1,2,4,7,2,4,...7,8\\
\bigskip\bigskip
Cyclomatic Complexity = 4
\end{column}
\end{columns}
\end{frame}



%
%%%%%%%%%%%%%%%%%%%%%%%%%%%%%%%%%%
%
%\begin{frame}[fragile]
%\frametitle{Example: Seat Reservation Program}
%\small
%\begin{verbatim}
%PROCEDURE SeatingPlan; 
%BEGIN 
    %If seatingplanlist.updated = new  then 
    %BEGIN 
        %While (sitzplanliste <> nil)  do 
        %BEGIN   
            %Case seatingplanlist.seat.status of 
              %CFree: seat_print (free); 
              %CUsed: seat_print (used); 
              %CReserved: seat_print (reserved); 
            %END; 
            %seatingplanlist := seatingplanlist.next; 
        %END;
        %seatingplanlist.updated := updated; 
    %END 
    %Else 
        %errorvar := not_new,
%END;
%\end{verbatim}
%\end{frame}
%
%%%%%%%%%%%%%%%%%%%%%%%%%%%%%%%%%%
%
%\begin{frame}
%\frametitle{Control flow chart for seat reservation}
%\begin{center}
%\dgrafik{.8\textheight}{.7\textwidth}{QualityControl/illustrations/SeatReservation}
%\end{center}
%\end{frame}

%%%%%%%%%%%%%%%%%%%%%%%%%%%%%%%%%

%\begin{frame}
%\frametitle{Test cases for edge coverage}
%\begin{tabular}{llp{0.2\textwidth}p{0.18\textwidth}p{0.28\textwidth}}
%\textbf{Nr} & \textbf{Class} & \textbf{Test case description} & \textbf{Expected Results}  & \textbf{Exemplary input data} \\
%1           & Normal          & Seating plan is new                    & Print seating plan        & New seating plan with one free, one occupied and one reserved seat must be passed as parameter. \\
            %& \multicolumn{4}{l}{Checked edges: 1, 4, 5, 6, 7, 8, 9, 10, 11, 12, 13} \\[2em]
%2           & Error          & Seating plan is old                    & Set the error variables    & Seating plan has not been changed since last use \\
            %& \multicolumn{4}{l}{Checked edges: 2, 3}
%\end{tabular}
%\end{frame}

%%%%%%%%%%%%%%%%%%%%%%%%%%%%%%%%%

\begin{frame}
\frametitle{Notes on the Edge-coverage criterion}
\begin{itemize}
  \item It is enforced, that every condition is passed with true and false.
  \item The edge coverage criterion is thus \emph{stronger} than the statement coverage criterion.
  \item Occasionally however not strong enough:
    \begin{itemize}
      \item Compound conditions are not sufficiently considered.
    \end{itemize}
  \item Insufficient for testing loops
\end{itemize}
\end{frame}

%%%%%%%%%%%%%%%%%%%%%%%%%%%%%%%%%

\begin{frame}[fragile]
\frametitle{Path (Branch) Coverage}
\framesubtitle{C2: Path (Branch) Coverage}
 Even under simple conditions, the Edge-coverage can still be insufficient: Testing of loops
\begin{columns}
\begin{column}{0.3\textwidth}
\lstset{language=Java,basicstyle=\small}
\begin{lstlisting}
if (x != 0) {
   y = 5;
}
else {
   y = 6;
}
if (z > 1) {
   z = z / x;
}
else {
   z = 0;
}
\end{lstlisting}
\end{column}
\pause
\begin{column}{0.7\textwidth}
\begin{itemize}
  \item The test set \{(x=0, z=1), (x=1, z=3)\} meets the Edge-coverage criterion, but does not recognize the division by 0.
  \item Dependencies remain undetected!
\end{itemize}
\end{column}
\end{columns}
\end{frame}

%%%%%%%%%%%%%%%%%%%%%%%%%%%%%%%%%

\begin{frame}
\frametitle{Edge-coverage (continued)}
\begin{itemize}
  \item Die Testmenge\\
   \{(x=0, z=1), (x=1, z=3), (x=0, z=3), (x=1, z=1)\}\\
   meets the Path-coverage criterion.
  \item Problem: The number of paths grows exponentially with the length of the program code, so testing with the Path-coverage criterion is unrealistic.
  \item Another problem: Number of paths for loops..\\
  Solution: Heuristics.\\
   Example heuristics for the number of loop cycles:
    \begin{itemize}
      \item 0 times
      \item \emph{moderately } often 
      \item maximum count
    \end{itemize}
  Boundary values are important!
\end{itemize}
\end{frame}

%%%%%%%%%%%%%%%%%%%%%%%%%%%%%%%%%

\begin{frame}
\frametitle{Condition coverage criterion}
\framesubtitle{C3}
Here e.g.\\
%\begin{lstlisting}
\begin{quote}
\texttt\sffamily{if (C1 and C2) ST;\\
~~else SF;}
\end{quote}
is transformed into the following statement:\\
\begin{quote}
\texttt\sffamily{if (C1)\\
~~if (C2) ST\\
~~else SF\\
else SF}
\end{quote}

\vspace{\baselineskip}
\begin{itemize}
  \item The new control flow graph covers all basic (sub)conditions and thus makes \emph{hidden} edges visible.
  \item The condition coverage criterion may therefore result in even stronger test sets.
\end{itemize}
\end{frame}
