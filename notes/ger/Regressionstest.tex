\mysubsubsection{Regressionstest}

%%%%%%%%%%%%%%%%%%%%%%%%%%%%%%%%%

\begin{frame}
\frametitle{Regressionstest}
\begin{itemize}
  \item Testwerkzeug speichert alle durchgef�hrten Testf�lle 
  \item Erlaubt die automatische Wiederholung aller bereits durchgef�hrten Tests nach �nderungen des Pr�flings 
  \item F�hrt Soll/Ist-Ergebnisvergleich durch
    \begin{itemize}
      \item Eingabedaten in das Testobjekt
      \item Erwartete Ausgabedaten oder Ausgabereaktionen (Soll-Ergebnisse)
    \end{itemize}
\end{itemize}
\end{frame}

%%%%%%%%%%%%%%%%%%%%%%%%%%%%%%%%%

%\begin{frame}
%\frametitle{Test Driven Development}
%\framesubtitle{Testgetriebene Entwicklung}
    %\begin{itemize}
    %\item Martin Fowler:\\
        %``Whenever you are tempted to type something into a print statement or a debugger expression, write it as a test instead.''
    %\item ``One of the ironies of Test Driven Development is that it isn't a testing technique. It's an analysis technique, a design technique, really a technique structuring all the activities of development.'' \cite{Beck2002}.                  
          %
    %\item Testen als Spezifikation und Entwurfsmittel.
          %
    %\item Tests sind eine Form von ausf�hrbarer Dokumentation.
          %
    %\item Regressionstests sind ein n�tzlicher Nebeneffekt.
          %
    %\end{itemize}
%\end{frame}

%%%%%%%%%%%%%%%%%%%%%%%%%%%%%%%%%

%\begin{frame}
%\frametitle{Werkzeugunterst�tzung zum Testen}
%\framesubtitle{Unit- und Integrations-Tests}
%\begin{itemize}
  %\item JUnit ist ein Testwerkzeug f�r Java \citep{Link2002}\\
    %(\url{http://www.junit.org/}).
	%\item Selenium-Integrations-Tests f�r Browser (\url{http://seleniumhq.org/}).
	%\item testIT WebTester als UI-Testautomatisierungs-Framework f�r Web-Applikationen basierend auf Selenium (\url{https://www.novatec-gmbh.de/produkte/testit-webtester/}).
%\end{itemize}
%Unit-Tests und Integrations-Tests sind mit diesen Werkzeugen automatisch wiederholbar.
%\end{frame}

%%%%%%%%%%%%%%%%%%%%%%%%%%%%%%%%%

%\begin{frame}
%\frametitle{Automatisierte Akzeptanztests}
%\framesubtitle{\citep{FIT2005,FIT2006}}
%\begin{itemize}
  %\item Die zu pr�fende Gesch�ftslogik und die erwarteten
    %Testergebnisse werden in Tabellenform dargestellt (HTML), so dass auch
    %Fachexperten an der Erstellung und Durchf�hrung der Tests
    %teilnehmen k�nnen.
  %\item FIT (Framework for Integrated Test) ist ein Werkzeug,
    %welches automatisierte und wiederholbare Akzeptanztests erlaubt
    %(\url{http://fit.c2.com/})
	%\item FitNesse erg�nzt FIT u.a.\ um ein Wiki zur kollaborativen Erstellung von Akzeptanztests (\url{http://www.fitnesse.org/})
%\end{itemize}
%\end{frame}

%%%%%%%%%%%%%%%%%%%%%%%%%%%%%%%%%

\begin{frame}
\frametitle{Testgetriebene Entwicklung und Refactoring}
\begin{itemize}
  \item Testgetriebene Entwicklung besteht aus der Kombination von zwei Techniken: 
    \begin{itemize}
      \item Testgetriebene Programmierung zur externen Evolution,
      \item Refactoring zur internen Evolution.
    \end{itemize}
  \item Refactoring\\
        Programm-�nderungen zur Verbesserung der internen Struktur, ohne die Funktionalit�t zu �ndern.
  \item Ein Refactoring ist dann erlaubt, wenn alle Tests erf�llt sind.
  \item Tests sind hier auch eine Form von ausf�hrbarer Anforderungsspezifikation.
\end{itemize}
\end{frame}

%%%%%%%%%%%%%%%%%%%%%%%%%%%%%%%%%

\begin{frame}
\frametitle{Continuous Integration}
Aufgaben, u.a.:
\begin{itemize}
	\item Ausf�hrung statischer Analysen (z.B.\ SpotBugs, Checkstyle)
	\item Bauen des Systems (z.B. Gradle)
	\item Automatisierte Ausf�hrung von Tests (z.B.\ JUnit, Selenium) \citep{TestAutomation2013}
	\item �berpr�fung von Log- und Monitoring-Daten (z.B.\ Kieker) \citep{SEN2015}
\end{itemize}
Siehe Gastvorlesung von Christian Zirkelbach, Alexander Krause und Marcel Bader.
\end{frame}

%%%%%%%%%%%%%%%%%%%%%%%%%%%%%%%%%

\begin{frame}
\frametitle{DevOps: Entwicklung und Betrieb}
\framesubtitle{Continuous Testing, Delivery \&\ Deployment}
\begin{center}
\pgfimage[width=0.7\textwidth]{Qualitaetssicherung/abbildungen/DevOps2016}
\end{center}
\end{frame}

%%%%%%%%%%%%%%%%%%%%%%%%%%%%%%%%%

\begin{frame}
\frametitle{Agilit�t und Zuverl�ssigkeit}
\framesubtitle{Durch Continuous Testing, Beispiel otto.de \citep{ICSA2017}}
\begin{center}
\pgfimage[width=\textwidth]{Qualitaetssicherung/abbildungen/DeploymentIncidentsBars}
\end{center}
\end{frame}
