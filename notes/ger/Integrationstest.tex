\mysubsubsection{Integrationstest}

%%%%%%%%%%%%%%%%%%%%%%%%%%%%%%%%%

\begin{frame}
\frametitle{Integrationstest}
\framesubtitle{Testing in the large}
\begin{itemize}
  \item Der Test einzelner Module garantiert nicht, dass die Zusammenarbeit korrekt funktioniert.
  \item Viele Module k�nnen gar nicht isoliert getestet werden.
  \item Also: Vorgehensweise zum Test komplexer Systeme (modulares Testen):
    \begin{itemize}
      \item Modultests (a.k.a.\ Unit Tests \&\ Components Tests)\\
			Testen eine Komponente ohne Kontext.
      \item Inkrementeller Integrationstest mehrer Komponenten im Kontext.
      \item Systemtest: Test des gesamten Systems in der Anwendungsumgebung (end-to-end).
    \end{itemize}
	\item Akzeptanztests (a.k.a.\ Funktionstests)\\
Focus auf das Testen von `cross-cutting' Funktionalit�t.
\end{itemize}
\end{frame}

%%%%%%%%%%%%%%%%%%%%%%%%%%%%%%%%%

\begin{frame}
\frametitle{Integrationstest (Forts.)}
\begin{itemize}
  \item Bei eingebetteten Realzeitsystemen muss auch das Zusammenspiel von Hard- und Software getestet werden.
	\begin{itemize}
		\item Digital Twins
	\end{itemize}
  \item Inkrementelles Testen ist dem \glq Big-Bang\grq-Testen, wobei nach den Modultests direkt das Gesamtsystem getestet wird, vorzuziehen.
  \item Die Trennung zwischen Schnittstelle und Implementierung erleichtert den Integrationstest erheblich.
    \begin{itemize}
      \item Leichtes Ersetzen von \glq Mockups\grq\ durch \glq richtige\grq\ Module.\\
			Siehe z.B.\ Mockito \url{https://github.com/mockito/}
    \end{itemize}
\end{itemize}
\end{frame}

%%%%%%%%%%%%%%%%%%%%%%%%%%%%%%%%%

\begin{frame}
\frametitle{Inkrementelles Testen}
\begin{center}
\pgfimage[width=0.9\textwidth]{Qualitaetssicherung/abbildungen/Inkrementelles_Testen}\\
"`Testger�st"'
\end{center}
 
\begin{itemize}
	\item Inkrementelle Tests k�nnen entsprechend den Benutzt- und Kompositions-Hierarchien bottom-up oder top-down erfolgen; nat�rlich auch jo-jo. 
	\item Eine hierarchische Architektur ist dabei sehr f�rderlich.
\end{itemize}
\end{frame}
