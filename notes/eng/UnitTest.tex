\mysubsection{Continuous integration}\label{ssec:ci}

%%%%%%%%%%%%%%%%%%%%%%%%%%%%%%%%%

\begin{frame}
 \frametitle{New features: \glqq Naive\grqq\ Workflow}
 
Typical  \glqq unstructured \grqq\ approach:
\begin{itemize}
   \item Implementation of new features
   \item Call and test of features from main method, output with \textbf{System.out.println}
   \item Manual check whether results are correct (or at least plausible)
   \item Test code in Main method is not used later
\end{itemize}
 
 \pause
 
Cons?
 \begin{itemize}
	 \item Code in Main method not  \glqq belonging there\grqq
		\item Test code is discarded, although it is still useful later
		\item Tests are no longer executed
   \item[$\rightarrow$] Regressions are not recognized
\end{itemize}
\end{frame}

%%%%%%%%%%%%%%%%%%%%%%%%%%%%%%%%%

\begin{frame}
\frametitle{Continuous integration}
Why?
\begin{itemize}
  \item Integration errors are continuously identified
  \item Early warnings for non-matching components
  \item Regression tests identify errors
  \item The developers are "`trained"' to be responsible
  \item Tool example:
  \begin{itemize}
	\item Jenkins \url{http://jenkins-ci.org/}
  \end{itemize}
\end{itemize}

\vfill

See also the lectures on continuous integration for Kieker/ExplorViz and at PPI.
\end{frame}

%%%%%%%%%%%%%%%%%%%%%%%%%%%%%%%%%

\begin{frame}
\frametitle{Continuous Integration Environment}
\framesubtitle{\tiny Continuous Integration in Zeiten agiler Programmierung https://heise.de/-1427092}
  \begin{center}
  \dgrafik{\textwidth}{0.3\textwidth}{ConfigurationManagement/image/ContinuousIntegrationEnvironment}
  \end{center}
\end{frame}

%%%%%%%%%%%%%%%%%%%%%%%%%%%%%%%%%

\mysubsubsection{Automated Testing}

%%%%%%%%%%%%%%%%%%%%%%%%%%%%%%%%%

\begin{frame}
 \frametitle{Automated Testing}
 
 Automatic tests:
  \begin{itemize}Test code in a specially designated area of the project
		\item Execution like \glqq Main methods\grqq, but tool support for:
		\begin{itemize}
    \item Comparison with expected behavior
    \item automatic execution of all tests
    \item Statistics on completed and failed tests
   \end{itemize}

\end{itemize}

 \pause
 
Pros:
\begin{itemize}
   \item Separation of functionality and tests
   \item Tests are retained
   \item Tests can be run continuously, automatically\\
      $\rightarrow$ Regressions are identified
   \item Tests as \glqq Qquality gateway\grqq\ for integration of new code into the master branch or master repository
\end{itemize}
\end{frame}

%%%%%%%%%%%%%%%%%%%%%%%%%%%%%%%%%

\begin{frame}
 \frametitle{Automated Testing: JUnit}
 
 \includegraphics[width=0.8cm]{ConfigurationManagement/image/junit.png}
\begin{itemize}
   \item First published 2000
   \item eclipse-support since 2002
   \item Integration e.g.~with Maven
\end{itemize}
\pause
 
\includegraphics[width=0.8cm]{ConfigurationManagement/image/junit.png} in Softwaretechnik:
\begin{itemize}
   \item Structure and example in lecture
   \item Details in exercise
   \item Your submission: JUnit-Tests required (Quality-Gateway for correction)
\end{itemize}
%https://www.vogella.com/tutorials/JUnit/article.html
\end{frame}

%%%%%%%%%%%%%%%%%%%%%%%%%%%%%%%%%

\begin{frame}[containsverbatim]{An Demo-function}
\framesubtitle{that should be tested \dots}
\lstset{language=Java}
\begin{lstlisting}[frame=single]
package com.example.project;

public class Calculator {

	public int add(int a, int b) {
		return a - b;
	}

}
\end{lstlisting}
\begin{flushright}
Take a look at this in eclipse.
\end{flushright}
\end{frame}

%%%%%%%%%%%%%%%%%%%%%%%%%%%%%%%%%

\begin{frame}[containsverbatim]{An Demo-function}
\footnotesize
\lstset{language=Java}
\begin{lstlisting}[frame=single]
package com.example.project;

import static org.junit.jupiter.api.Assertions.assertEquals;
import org.junit.jupiter.api.DisplayName;
import org.junit.jupiter.api.Test;

class MyTest {

	@Test
	@DisplayName("1 + 1 = 2")
	void addsTwoNumbers() {
	   Calculator calculator = new Calculator();
	   assertEquals(2, calculator.add(1, 1),
		                "1 + 1 should equal 2");
	}

}
\end{lstlisting}
%import org.junit.jupiter.params.ParameterizedTest;
%import org.junit.jupiter.params.provider.CsvSource;
	%@ParameterizedTest(name = "{0} + {1} = {2}")
	%@CsvSource({
			%"0,    1,   1",
			%"1,    2,   3",
			%"49,  51, 100",
			%"1,  100, 101"
	%})
	%void add(int first, int second, int expectedResult) {
		%Calculator calculator = new Calculator();
		%assertEquals(expectedResult, calculator.add(first, second),
				%() -> first + " + " + second + " should equal " + expectedResult);
	%}

\begin{flushright}
\tiny
\url{https://github.com/junit-team/junit5-samples/}
\end{flushright}
\end{frame}

%%%%%%%%%%%%%%%%%%%%%%%%%%%%%%%%%

\begin{frame}
 \frametitle{Automated Testing: Summary}
 
 Evaluation:
 \begin{itemize} Automatic tests cause few additional work!
		\item Help to recognize quality problems early.
\end{itemize}
 
\pause
 
Questions:
\begin{itemize}
   \item How many tests should I write?
   \item Which parts of the software should be tested?
   \item And what sample data?
   \item From where do I know the  \glqq expected\grqq\ results?
   \item \dots
\end{itemize}

\pause
 
More on this in the section \glqq Quality control\grqq.
\end{frame}

%%%%%%%%%%%%%%%%%%%%%%%%%%%%%%%%%

\mysubsubsection{Mocking}

%%%%%%%%%%%%%%%%%%%%%%%%%%%%%%%%

\begin{frame}[allowframebreaks]
 \frametitle{Mocking: Motivation}
 
Unit-Testing:
\begin{itemize}
   \item Code uses other objects of the system
   \item Objects must be accessible for testing
\end{itemize}
Issues:
 \begin{itemize}
	   \item Unit-Testing! No complete system run!
		 \item Other objects / classes may not be available yet!
		 \item Performance
    \item Difficult to provide real object:
    \begin{itemize}
     \item Specific parameters (time, sensor values)
     \item Error states (network error, \dots)
    \end{itemize}
\end{itemize}

 \pause 
Examples:
\begin{itemize}
   \item Time-Server
   \item Database
   \item Temperature-Sensor
   \item Network services
\end{itemize}

Mocking is particularly relevant for \glqq Test Driven Development\grqq\.
\end{frame}

%%%%%%%%%%%%%%%%%%%%%%%%%%%%%%%%%

\begin{frame}
 \frametitle{Approach: Mock objects}
 
Alternative: Mock objects (also: Test Doubles):
\begin{itemize}Objects that resemble real behavior
   \item Can be used in test instead of the real object.
\end{itemize}
 
\pause

Interface: 
\begin{itemize} Must be usable instead of the \glqq real object\grqq\ \\
		$\rightarrow$ Same interface (or subinterface) as real object
\end{itemize}
\pause
 
Behaviour:
\begin{itemize}\glqq Based on\grqq\ real object\\
		$\rightarrow$ for code to be tested: like real object
		\item Explicit triggering of values, states, \dots\\
   $\rightarrow$ for Test-Code:\\ configurable!
\end{itemize}
\end{frame}

%%%%%%%%%%%%%%%%%%%%%%%%%%%%%%%%%

\begin{frame}
 \frametitle{Mock objects: Differentiations}
 
Most simple version: Stub object
\begin{itemize}
  \item Always returns fixed results to requests
  \item Results can possibly be set by the test code
\end{itemize}

 \dots
 \pause
 
Most complex version: Fake object
\begin{itemize}
  \item Working implementation as real object
  \item Take shortcuts and behave in a much simpler way
\end{itemize}
\end{frame}

%%%%%%%%%%%%%%%%%%%%%%%%%%%%%%%%%

\begin{frame}
 \frametitle{Mock objects: Categories}
  \framesubtitle{\url{https://martinfowler.com/articles/mocksArentStubs.html}}
 
 \begin{description}[Dummy objects]
  \item[Dummy objects] are passed around but never actually used. Usually they are just used to fill parameter lists. \pause
  \item[Fake objects] actually have working implementations, but usually take some shortcut which makes them not suitable for production (an in-memory database is a good example). \pause
  \item[Stubs] provide canned answers to calls made during the test, usually not responding at all to anything outside what's programmed in for the test. \pause
  \item[Spies] are stubs that also record some information based on how they were called. \pause
  \item[Mocks] are [\dots] objects pre-programmed with expectations which form a specification of the calls they are expected to receive.
 \end{description}
\end{frame}

%%%%%%%%%%%%%%%%%%%%%%%%%%%%%%%%%

\begin{frame}
 \frametitle{Framework: Mockito}
 
�bersicht:
\begin{itemize}
	\item \url{https://site.mockito.org/}
   \item Java-Framework for creating Mock objects
   \item First release: 2008
\end{itemize}

If you use Mockito in tests you typically:
\begin{enumerate}
	\item Mock away external dependencies and insert the mocks into the code under test
	\item Execute the code under test
	\item Verify that the code executed correctly
\end{enumerate}
For example, you can verify that a method has been called with certain parameters. This kind of testing is sometimes called behavior testing. Behavior testing does not check the result of a method call, but it checks that a method is called with the right parameters.
\end{frame}

%%%%%%%%%%%%%%%%%%%%%%%%%%%%%%%%%

\begin{frame}[containsverbatim,allowframebreaks]
 \frametitle{Mockito with JUnit}
\footnotesize
\lstset{language=Java}
\begin{lstlisting}[frame=single]
import static org.mockito.Mockito.*;

public class MockitoTest  {

  @Mock
  MyDatabase databaseMock;                         // 1

  @Rule public MockitoRule mockitoRule
		            = MockitoJUnit.rule(); // 2

  @Test
  public void testQuery()  {
    ClassToTest t = new ClassToTest(databaseMock); // 3
    boolean check = t.query("* from t");           // 4
    assertTrue(check);                             // 5
    verify(databaseMock).query("* from t");        // 6
  }
}
\end{lstlisting}
\newpage
\normalsize
\begin{enumerate}
	\item Tells Mockito to mock the databaseMock instance
	\item Tells Mockito to create the mocks based on the \texttt{@Mock} annotation
	\item Instantiates the class under test using the created mock
	\item Executes some code of the class under test
	\item Asserts that the method call returned true
	\item Verify that the query method was called on the \texttt{MyDatabase} mock
\end{enumerate}

Siehe auch \url{https://www.vogella.com/tutorials/Mockito/article.html}
\end{frame}

%%%%%%%%%%%%%%%%%%%%%%%%%%%%%%%%%

\begin{frame}
 \frametitle{Mocking: Summary}
 
Motivation: \glqq Replace\grqq complex objects for test:
\begin{itemize}
  	\item Simulation
    \item Replace with dummies
    \item Collect additional debug information
\end{itemize}
Connection to Unit-Testing:
\begin{itemize}Extends assertions by Verification: Testing of interactions with mock objects
\end{itemize}
Verification:
\begin{itemize}
	\item Mock objects log function calls. This allows to check properties of the \textit{stack trace (call history)} of mock objects.
  \item \textbf{Note}: Do not mistake for formal verification!
\end{itemize}
Unit Tests \textit{vs.} Mock-Verification?
\begin{itemize}
	\item Addition, not replacement: Check different aspects!
  \item \textbf{More details and examples: practice!}
\end{itemize}
\end{frame}
