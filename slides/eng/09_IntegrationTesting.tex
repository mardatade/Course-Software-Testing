%%%%%%%%%%%%%%%%%%%%%%%%%%%%%%%%%

\section{Integration Testing}

%%%%%%%%%%%%%%%%%%%%%%%%%%%%%%%%%

\subsection{Integration test}

%%%%%%%%%%%%%%%%%%%%%%%%%%%%%%%%%

\begin{frame}
\frametitle{Integration testing}
\framesubtitle{testing in the large}
\begin{itemize}
  \item Testing individual modules separately does not guarantee a correct inter-operation of multiple modules.
  \item Many modules cannot be tested in isolation.
  \item Additionally: Steps for testing complex systems (modular testing):
    \begin{itemize}
      \item Modular tests (a.k.a.\ Unit Tests \&\ Components Tests)\\
			Testing a component without context.
      \item Incremental integration test of multiple components in their context.
      \item System test: Test of the whole system in the application environment (end-to-end).
    \end{itemize}
	\item Acceptance tests (a.k.a.\ Funktional tests)\\
Focus on testing of `cross cutting' functionality.
\end{itemize}
\end{frame}

%%%%%%%%%%%%%%%%%%%%%%%%%%%%%%%%%

\begin{frame}
\frametitle{Integration testing (contd.)}
\begin{itemize}
  \item For embedded real time systems, the interoperation of hardware and software needs to be tested.
	\begin{itemize}
		\item Digital Twins
	\end{itemize}
  \item Incremental testing should be preferred over \glq Big-Bang\grq testing, which directly goes for the full system right after unit testing.
  \item A clear separation of interface and implementation makes integration testing easy.
    \begin{itemize}
      \item Easier swapping of \glq mock-ups\grq\ for \glq real\grq\ modules.\\
			See, e.g., Mockito \url{https://github.com/mockito/}
    \end{itemize}
\end{itemize}
\end{frame}

%%%%%%%%%%%%%%%%%%%%%%%%%%%%%%%%%

\subsection{Incremental testing}

%%%%%%%%%%%%%%%%%%%%%%%%%%%%%%%%%
\begin{frame}
\frametitle{Incremental testing}
\begin{center}
\pgfimage[width=0.8\textwidth]{images/Qualitaetssicherung/abbildungen/Inkrementelles_Testen}\\
"`Test framework"'
\end{center}
 
\begin{itemize}
	\item Incremental tests can be bottom-up, top-down (or even jo-jo), with respect to the hierarchy of composition.
	\item A hierarchical architecture is very effective for this purpose.
\end{itemize}

\end{frame}

%%%%%%%%%%%%%%%%%%%%%%%%%%%%%%%%%

\subsection{Incremental testing Strategies}

%%%%%%%%%%%%%%%%%%%%%%%%%%%%%%%%%

\begin{frame}[fragile]
\framesubtitle{Strategies}

\begin{enumerate}[align=parleft]
\item [\small top-down:] Highest level module are tested and integrated first. Data flow is also tested early in the process.
\begin{itemize}
\item [Pro:] Limits driver
\item [Con:] Complicate as of stubs, low level units are tested late, no early release of functionality
\end{itemize}
\item [\small bottom-up:] Lowest level units are tested and integrated first.
\begin{itemize}
\item [Pro:] Minimize need for stubs
\item [Con:] Complicate as of drivers, high level logic and data flow is tested late, no early release of functionality
\end{itemize}
\item [\small Jo-Jo:] Testing along functional data and control flow path
\begin{itemize}
\item [] output functions are integrated in top-down
\item [] input functions are integrated in bottom-up
\item [Pro:] limited early release of functionality, minimize stubs and drivers
\end{itemize}
\end{enumerate}
\end{frame}