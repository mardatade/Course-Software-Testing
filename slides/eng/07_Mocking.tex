%%%%%%%%%%%%%%%%%%%%%%%%%%%%%%%%%

\section{Mocking}

%%%%%%%%%%%%%%%%%%%%%%%%%%%%%%%%%

\subsection{Categories}

%%%%%%%%%%%%%%%%%%%%%%%%%%%%%%%%%
\begin{frame}
 \frametitle{Mock objects: Categories}
 \footnotesize
 \begin{description}
  \item[Dummy objects] are passed around but never actually used. Usually they are just used to fill parameter lists.
  \item[Fake objects] actually have working implementations, but usually take some shortcut which makes them not suitable for production (an in-memory database is a good example).
  \item[Stubs] provide canned answers to calls made during the test, usually not responding at all to anything outside what's programmed in for the test.
  \item[Spies] are stubs that also record some information based on how they were called.
  \item[Mocks] are [\dots] objects pre-programmed with expectations which form a specification of the calls they are expected to receive.
 \end{description}
\end{frame}

%%%%%%%%%%%%%%%%%%%%%%%%%%%%%%%%%

\subsection{Summary}

%%%%%%%%%%%%%%%%%%%%%%%%%%%%%%%%%

\begin{frame}
 \frametitle{Mocking: Summary}
 
\begin{description}
\item [Motivation:] \alert{Replace} complex objects for test:
\begin{itemize}
  \item Simulation
  \item Replace with dummies
  \item Collect additional debug information
\end{itemize}
\item [Connection to Unit-Testing:] \alert{Extends assertions} by Verification. Testing of interactions with mock objects.
\item [Verification:]
\begin{itemize}
  \item [] Mock objects log function calls. This allows to check properties of the \textit{stack trace (call history)} of mock objects.
  \item [] \alert{No formal verification}
\end{itemize}
\item [Unit Tests \textit{vs.} Mock-Verification?] \alert{Addition, not replacement} but different aspects.
\end{description}
\end{frame}